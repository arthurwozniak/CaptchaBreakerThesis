%%%  Ukázkový text a dokumentace stylu pro text závěrečné (bakalářské a
%%%  diplomové) práce na KI PřF UP v Olomouci
%%%  Copyright (C) 2012 Martin Rotter, <rotter.martinos@gmail.com>
%%%  Copyright (C) 2014 Jan Outrata, <jan.outrata@upol.cz>


%%  Pro získání PDF souboru dokumentu je třeba tento zdrojový text v
%%  LaTeXu přeložit (dvakrát) programem pdfLaTeX.

%%  V případě použití programu BibLaTeX pro tvorbu seznamu literatury
%%  je poté ještě třeba spustit program Biber s parametrem jméno
%%  souboru zdrojového textu bez přípony a následně opět (dvakrát)
%%  přeložit zdrojový text programem pdfLaTeX.

%%  Postup získání Postscriptového souboru je popsán v dokumentaci.


%%  Třída dokumentu implementující styl pro závěrečnou práci. Vybrané
%%  nepovinné parametry (ostatní v dokumentaci):

%%  'master' pro sazbu diplomové práce, jinak se sází bakalářská práce

%%  'field=kód' pro Váš studijní obor, kódy pro diplomovou práci 'uvt'
%%  pro Učitelství výpočetní techniky pro střední školy a 'binf' pro
%%  Bioinformatiku, jinak je výchozí Informatika, a pro bakalářskou
%%  práci 'ainfk' pro Aplikovanou informatiku v kombinované formě,
%%  'inf' pro Informatiku, 'infv' pro Informatiku pro vzdělávání a
%%  'binf' pro Bioinfomatiku, jinak je výchozí Aplikovaná informatika
%%  v prezenční formě

%%  'printversion' pro sazbu verze pro tisk (nebarevné logo a odkazy,
%%  odkazy s uvedením adresy za odkazem, ne odkazy do rejstříku),
%%  jinak verze pro prohlížeč

%%  'biblatex' pro zapnutí podpory pro sazbu bibliografie pomocí
%%  BibLaTeXu, jinak je výchozí sazba v prostředí thebibliography

%%  'language=jazyk' pro jazyk práce, jazyky english pro anglický,
%%  slovak pro slovenský, jinak je výchozí czech pro český

%%  'font=sans' pro bezpatkový font (Iwona Light), jinak výchozí
%%  patkový (Latin Modern)

\documentclass[
%  master,
%  field=inf,
%  printversion,
  master=true,
  biblatex,
%  language=english,
%  font=sans,
  glossaries,
  index
]{kidiplom}

%% Informace pro úvodní strany. V jazyku práce (pokud není v komentáři
%% uvedeno česky) a anglicky. Uveďte všechny, u kterých není v
%% komentáři uvedeno, že jsou volitelné. Při neuvedení se použijí
%% výchozí texty. Text pro jiný než nastavený jazyk práce (nepovinným
%% parametrem language makra \documentclass, výchozí český) se zadává
%% použitím makra s uvedením jazyka jako nepovinného parametru.

%% Název práce, česky a anglicky. Měl by se vysázet na jeden řádek.
\title{Prolamování CAPTCHA zabezpečení}
\title[english]{Breaking the CAPTCHA}

%% Volitelný podnázev práce, česky a anglicky. Měl by se vysázet na
%% jeden řádek. Výchozí je prázdný.
% \subtitle{Ukázkový text a dokumentace stylu v \LaTeX{}u}
% \subtitle[english]{Sample text and documentation of the \LaTeX{} style}

%% Jméno autora práce. Makro nemá nepovinný parametr pro uvedení
%% jazyka.
\author{Bc. Kamil Hanus}

%% Jméno vedoucího práce (včetně titulů). Makro nemá nepovinný
%% parametr pro uvedení jazyka.
\supervisor{Mgr. Martin Trnečka, Ph.D.}

%% Volitelný rok odevzdání práce. Výchozí je aktuální (kalendářní)
%% rok. Makro nemá nepovinný parametr pro uvedení jazyka.
%\yearofsubmit{\the\year}

%% Anotace práce, včetně anglické (obvykle překlad z jazyka
%% práce). Jeden odstavec!
\annotation{Ukázkový text závěrečné práce na Katedře informatiky
  Přírodovědecké fakulty Univerzity Palackého v Olomouci, který je
  zároveň dokumentací stylu pro text práce v \LaTeX{}u. Zdrojový text
  v \LaTeX{}u je doporučeno použít jako šablonu pro text skutečné
  závěrečné práce studenta.}

\annotation[english]{Sample text of thesis at the \kitextdepten,
  \kitextfacultyen, \kitextuniven{} and, at the same time,
  documentation of the \LaTeX{} style for the text. The source text in
  \LaTeX{} is recommended to be used as a template for real student's
  thesis text.}

%% Klíčová slova práce, včetně anglických. Oddělená (obvykle) středníkem.
\keywords{styl textu; závěrečná práce; dokumentace; ukázkový text}
\keywords[english]{text style; thesis; documentation; sample text}

%% Volitelná specifikace příloh textu práce, i anglicky. Výchozí je '1
%% CD/DVD'.
%\supplements{jedno kulaté placaté CD/DVD s malou kulatou dírou uprostřed}
%\supplements[english]{one round flat CD/DVD with a small round hole in the middle}

%% Volitelné poděkování. Stručné! Výchozí je prázdné. Makro nemá
%% nepovinný parametr pro uvedení jazyka.
\thanks{Děkuji, děkuji, děkuji.}

%% Cesta k souboru s bibliografií pro její sazbu pomocí BibLaTeXu
%% (zvolenou nepovinným parametrem biblatex makra
%% \documentclass). Použijte pouze při této sazbě, ne při (výchozí)
%% sazbě v prostředí thebibliography.
\bibliography{bibliografie.bib}

%% Další dodatečné styly (balíky) potřebné pro sazbu vlastního textu
%% práce.
\usepackage{lipsum}
\usepackage{verbatimbox}

\begin{document}
%% Sazba úvodních stran -- titulní, s bibliografickými údaji, s
%% anotací a klíčovými slovy, s poděkováním a prohlášením, s obsahem a
%% se seznamy obrázků, tabulek, vět a zdrojových kódů (pokud jejich
%% sazba není vypnutá).
\maketitle

%% Vlastní text závěrečné práce. Pro povinné závěry, před přílohami,
%% použijte prostředí kiconclusions. Povinná je i příloha s obsahem
%% přiloženého CD/DVD.

%% -------------------------------------------------------------------

\newcommand{\BibLaTeX}{\textsc{Bib}\LaTeX}



\section{Úvod}
\newpage
\section{Captcha}
\subsection{Historie}
 - důvody vzniku, první captcha, vývoj
\subsection{Druhy a možnosti prolamování}
 - textová, obrázková, audio, noCaptcha, aka slider na binance
 \subsection{Strojové učení}
 - NN a CNN
\newpage
\section{Captcha Breaker}
V rámci diplomové práce byla vyvinuta webová aplikace nazvaná CaptchaBreaker. Po prostudování různých přístupů ke tvorbě CAPTCHA schémat a možností jejich prolamování byla vybrána skupina CAPTCHA obrázků pro další studium. Cílem je vytvořit aplikaci, ve které je možné vytvořit dataset CAPTCHA obrázků, zadefinovat způsob prolamování konkrétního schématu a posléze veřejnosti poskytnout nástroj na prolamování konkrétního schématu.

\subsection{Použité technologie}
\subsubsection*{Flask}
Flask je microframework určený pro tvorbu webových aplikací napsaný v jazyce Python. Samotné jádro frameworku v základu obsahuje s nadsázkou pouze nástroje pro směrování požadavků a šablonovací systém Jinja. Vyžadujeme-li funkcionalitu navíc, jako například práci s DB, ORM mapování, validaci formulářů nebo autentizaci požadavků, je nutné doinstalovat patřičný modul. Framework se tedy snaží být minimalistický a je pouze na uvážení vývojáře, jaké moduly ke strohému jádru dodá. 
\subsubsection*{PostgreSQL}
Pro perzistentní uložení informací bylo nutné zvolit některý z relačních databázových enginů. V tomto případě padla volba na PostgreSQL, což je open-source relační SQL databáze. Stejně tak by pro potřeby projektu byla vyhovující jakýkoliv Flaskem podporovaný engine, například MariaDB nebo MySQL.
\subsubsection*{Celery}
Důležitým vlastností pro některé aplikace je možnost vykonávat některé operace asynchronně. K tomu lze v případě Flasku použít například Celery. Jedná se distribuovaný systém sloužící pro zpracování zasílaných zpráv, které předávány skrze nějakého z podporovaných prostředníků, tzv. brokerů. I když primárně míří na zpracování v reálném čase, podporuje například zařazení obdržených zpráv do fronty a jejich sekvenční zpracování.

Chceme-li vytvořit nějakou metodu, kterou bude možné v pythonu spustit asynchronně, je nutné obalit ji dekorátorem \texttt{@celery.task}. Tím je umožněno instanci Celery workeru vyhledat všechny tasky v dané aplikaci a uložit si je do paměti. Obdrženou zprávu lze poté chápat jako dvojici \texttt{název metody}-\texttt{argumenty}. Pracujeme-li s databázovými objekty, je možné k jejich předávání přistoupit dvěma způsoby. Buď jako argument předáme ID objektu a v metodě samotné je objekt načítán z databáze a nebo předáme jako argument objekt samotný, který je následně serializován před zasláním zprávy.

\subsubsection*{Redis}
Jako broker pro komunikaci s Celery byl použit Redis. Jedná se o úložiště pracující primárně v operační paměti, nicméně lze jej nakonfigurovat tak, aby ukládal data na pevný disk a po restartu zařízení byl schopen obnovit svůj stav.

\subsubsection*{jQuery}
jQuery je již několik let nejpoužívanější javascriptovou knihovnou \footnote{$https://w3techs.com/technologies/overview/javascript_library/all$} vůbec. Mezi její silné stránky patří snadná manipulace s DOM nebo tvorba AJAX requestů. Těchto vlastností využívá zejména průvodce tvorbou datasetu v administrační části aplikace.
\subsubsection*{Bootstrap}
Spolu s jQuery tvoří CSS knihovna Bootstrap základ velkého množství webů. Programátor je s jejím použítím odstíněn od nutnosti stylovat prvky webové stránky a je možné se více soustředit na samotný vývoj backendu.
\subsubsection*{PyTorch}
Open-source knihovna zaměřená na strojové učení v jazyce Python. 


\subsection{Extrakce symbolů}
 - obecné popsání algoritmu na extrakci symbolů

\subsection{Uživatelská část}
Část aplikace, která je dostupná široké veřejnosti, lze rozdělit do dvou skupin -- prezentace známých schémat a API. Domovská stránka seznamuje uživatele zejména s účelem projektu a dává mu možnost vyzkoušet si prolamování CAPTCHA obrázků v praxi. Po výběru obrázku k prolomení a příslušného klasifikátoru je možné zaslat dotaz na server, který z obrázku dekóduje znaky pomocí vybraného klasifikátoru, což je řešeno asynchronním dotazem na API rozhraní. Nastane-li chyba, je uživateli zobrazena příslušná hláška. V případě úspěšného rozpoznání pouze obsah zaslaného obrázku.\\

API obsahuje pouze jednu cestu, na kterou se zasílá POST request se dvěma parametry -- ID klasifikátoru a obrázek zakódovaný do Base64. Jakmile server obdrží požadavek, nejprve zkontroluje existenci klasifikátoru. Pokud přijal špatnou hodnotu, informuje o tom v odpovědi. Následně dekóduje obrázek, zjistí jeho formát (jpeg, png, gif, etc..) a normalizuje jej do podoby kterou akceptuje klasifikátor, resp. příslušný proces extrakce symbolů.

\begin{minipage}{0.45\textwidth}
\begin{verbatim}
{
"message":
"Unknown classifier",
"status":
"error"
}
\end{verbatim}
Chybová odpověď
\end{minipage}
\hfill
\begin{minipage}{0.45\textwidth}
\begin{verbatim}
{
"message":
"10487",
"status":
"success"
}
\end{verbatim}
Odpověď v případě rozpoznání znaků
\end{minipage}

\begin{table}
\begin{tabular}{|l|l|l|}
\hline
\textbf{URL} & \textbf{metoda} & \textbf{parametry}
\\ \hline
\texttt{/} & \texttt{GET} & -
\\ \hline
\texttt{/decode/} & \texttt{POST} & image, classifier ID
\\ \hline
\end{tabular}
\caption{URL endpointy pro globální namespace}
\end{table}

\subsection{Administrační rozhraní}
Část webu dostupná pouze administrátorovi je dostupná na URL \texttt{/admin}. Všechny dotazy přicházející na adresy v tomto jmenném prostoru musejí být autentizovány. Jelikož se nepředpokládá užívání aplikace více administrátory, jsou údaje poskytnuté na přihlašovací stránce kontrolovány pouze vůči hodnotám \texttt{APP\_USERNAME} a \texttt{APP\_PASSWORD} v konfiguračním souboru. Bezpečnost takového přístupu lze dále zvýšit například restrikcí IP adres u dotazů přicházející na tyto adresy v konfiguraci použitého aplikačního serveru.

\subsubsection*{Nahrání datasetu \texttt{/admin/datasets/new/}}
Proces nahrání datasetu obsahuje konfigurátor vytvořený pomocí knihoven jQuery a  Bootstrap. Administrátor je nejprve vyzván k vybrání ZIP archivu se vzorovými CAPTCHA obrázky. Následně je mu umožněno označit nahrané obrázky texty, které obsahují. V posledním kroku je konfigurátor operací, díky nimž je možné zadefinovat pro nahrané schéma postup odstranění šumu pro extrahování symbolů. Dostupné jsou zejména tzv. morfologické operace, nicméně aplikace je navržena univerzálně a po vytvoření nové třídy reprezentující operaci dědící z třídy \texttt{AbstractOperation} je možné výběr rozšířit. Jakmile je administrátor spokojený s výsledkem, odešle dataset na server a je přesměrován na stránku s jeho detaily.
\subsubsection*{Zobrazení datasetu \texttt{/admin/datasets/:id/}}
Jakmile je dataset nahrán na server, je administrátor přesměrován na stránku s jeho detaily. V horním řádku stránky jsou informace o času vytvoření, znaky obsažené v datasetu a celkový počet rozpoznaných symbolů. Následuje výpis nahraných obrázků spolu se zobrazením extrahovaných symbolů. V poslední části je vypsána konfigurace extraktoru ve formátu JSON. 

\subsubsection*{Tvorba klasifikátoru \texttt{/admin/classifiers/new/}}
Formulář konfigurace klasifikátoru je parametrizován čtyřmi vstupy - název klasifikátoru, maximální počet iterací, cílová hodnota ztrátové funkce a zdrojový dataset. Po odeslání požadavku na vytvoření klasifikátoru je samotná fáze trénování spuštěna na pozadní a administrátor je přesměrován na nástěnku, kde vidí průběh trénování.

\subsubsection*{Detaily klasifikátoru \texttt{/admin/classifier/:id/}}
Stránka zobrazující detaily klasifikátoru v současné době zobrazuje informace z fáze učení klasifikátoru. Tedy kromě názvu klasifikátoru a zvoleného datasetu jde o výslednou hodnotu ztrátové funkce a stav fáze učení klasifikátoru. Kromě toho je možné již nahraný dataset odstranit.

\subsubsection*{Nástěnka \texttt{/admin/overview/}}
Jelikož se prvky nástěnky odkazují na pojmy vyjmenované výše, je popis domovské stránky administrátora až poslední. Na nástěnce jsou zobrazeny statistiky příchozích dotazů na rozpoznání CAPTCHA obrázků a počtu datasetů, resp. klasifikátorů. Existují-li navíc klasifikátory, které se momentálně trénují, jsou na nástěnce zobrazeny informace o stavu procesu. Kromě názvu klasifikátoru se jedná o pořadí současné iterace fáze učení nebo aktuální hodnotu ztrátové funkce. Tyto informace se pravidelně obnovují každých 5 sekund po asynchronním vyžádání detailů. Jakmile je klasifikátor natrénován, záznam z nástěnky zmizí.

\begin{table}[h]
\begin{tabular}{|l|l|l|}
\hline
\textbf{URL} & \textbf{metoda} & \textbf{parametry}
\\ \hline
\texttt{/} & \texttt{GET} & -
\\ \hline
\texttt{/overview/} & \texttt{GET} & -
\\ \hline
\texttt{/datasets/} & \texttt{GET} & -
\\ \hline
\texttt{/datasets/new/} & \texttt{GET, POST} & - 
\\ \hline
\texttt{/datasets/new/preview} & \texttt{POST} & image, operations
\\ \hline
\texttt{/datasets/:id/} & \texttt{GET} & -
\\ \hline
\texttt{/datasets/:id/delete/} & \texttt{POST} & -
\\ \hline
\texttt{/classifiers/} & \texttt{GET} & -
\\ \hline
\texttt{/classifiers/new/} & \texttt{GET, POST} & - 
\\ \hline
\texttt{/classifiers/:id/} & \texttt{GET} & -
\\ \hline
\texttt{/classifiers/:id/delete/} & \texttt{POST} & -
\\ \hline
\texttt{/task\_status/:id/} & \texttt{GET} & -
\\ \hline
\end{tabular}
\caption{URL endpointy pro namespace \texttt{/admin/}}
\end{table}


\subsection*{Další vývoj}
Jak je patrné již z popisu aplikace, k vytvoření ultimátního nástroje pro lámání CAPTCHA kódů, byť pouze obrázkových, vede ještě dlouhá cesta. Implementace následujících funkcionalit, zejména v administračním rozhraní, by výrazně zlepšila komfort užívání a zřejmě i praktičnost systému jako celku.

\begin{itemize}
\item \textbf{Upozornění na chybně klasifikované znaky.} Na stránce zobrazující detaily klasifikátoru se naskýtá možnost zobrazit chybně klasifikované obrázky. Taková chyba může nastat zejména v následujících třech případech: 1) Příliš podobné znaky (číslice nula versus písmeno O). 2) Algoritmus pro extrakci symbolů na výstupu nevrátí celý znak, ale např. pouze jeho část. 3) Symbol byl ručně chybně označen a klasifikátor jej rozpoznává správně.
\item \textbf{Změna označení již nahraných obrázků.} Během procesu vytváření datasetu musí administrátor nahrát ZIP archiv obsahující obrázky jejichž název odpovídá CAPTCHA kódu, nebo je procesem označení obrázků proveden před uploadem na server. V obou případech však závisí jen a pouze na lidském faktoru, zda bude obrázku přiřazen chybný text. To je například poslední bod předchozího odstavce. Z toho důvodu je vhodné umožnit změnu označení obrázku, resp. samotného symbolu.
\item \textbf{Odstranění některých znaků}. Druhá chyba z prvního odstavce popisuje nesprávné rozpoznání symbolu v CAPTCHA obrázku. Tomu je možné předejít úpravou parametrů extraktoru, což však může mít za následek horší rozpoznávací schopnosti symbolů v datasetu jako celku. Abychom zamezili trénování klasifikátoru na chybných datech, je vhodnější poskytnout možnost odebrat jednotlivé symboly z datasetu, které byly špatně rozpoznány. Přímý důsledek tohoto řešení je nutná úprava procesu trénování klasifikátoru CNN, resp. hodnoty udávající počet současně zpracovávaných vstupů.
\item  \textbf{Volba z více klasifikátorů.} V současné době systém používá jako klasifikátor konvoluční neuronovou síť se dvěma konvolučními vrstvami, kde se na každý vstup uplatní funkce \texttt{MaxPool} s velikostí jádra = 2. To nemusí vyhovovat všem rozpoznávaným instancím CAPTCHA obrázků a i pro účely porovnání efektivnosti klasifikátorů je rozumné přidat další a mít možnost volby. 
\item \textbf{Volba tvaru jádra morfologických operací.} Každá morfologická operace potřebuje dva vstupy - obrázek a jádro. Tvar, resp. hodnota matice, jádra má přímý vliv na výsledný obrázek. Aplikace parametrizuje jádro pouze jednou celočíselnou hodnotou udávající velikost čtvercové jednotkové matice. Je však možné použít jádro ve tvaru tzv. kříže, resp. elipsy. Příklad lze vidět na obrázku níže.
\item \textbf{Klasifikátor používající více datasetů.} K dalšímu zlepšení rozpoznávání by mohlo také vést trénování klasifikátoru znaky z více zdrojů. Kromě primárního datasetu, který udává i proces extrakce symbolů, by k jeho množině znaků byly přidány ty symboly s odpovídající hodnotou ze zvolených datasetů. Tím by se docílilo rozpoznání transformací, které v trénovací množině symbolů nebyly prve zahrnuty.
\end{itemize}


Otázky\\
\begin{itemize}
\item n-fold u klasifikátoru? s tím souvisí úspěšnost klasifikátoru
\item experimentální porovnání -- nn vs cnn, rozpoznání na úrovni obrázků vs znaků
\item počet stran práce
\item u datasetu zobrazit pokrytí jednotlivých symbolů
\item paginace obrázků datasetu

\end{itemize}
\newpage
\section{Styly pro psaní bakalářských a diplomových prací}
Toto jsou styly pro psaní bakalářských a diplomových prací přes typografický systém \LaTeX{}, tedy \textbf{kistyles}.

\subsection{Požadavky a podprovaná prostředí}
Sada balíku \textbf{kistyles} podporuje následující distribuce systému \LaTeX{}:
\begin{itemize}
\item \TeX{} Live.
\end{itemize}

Jsou podporovány všechny výstupní ovladače, tedy jak \textbf{dvi}, tak \textbf{pdf} i \textbf{ps}. Funkčnost zmiňovaných distribucí byla ověřena na několika operačních systémech, mezi které patří:
\begin{enumerate}
\item Windows $8.1$,
\item Archlinux,
\item Debian.
\end{enumerate}

Důrazně se doporučuje používat aktuální verzi dané distribuce systému \LaTeX{}.

%%%  Po přeložení programem CSLaTeX (třikrát) je potřeba použít
%%%  program DVIPS a takto získaný PostScriptový soubor vytisknout
%%%  na PostScriptové tiskárně nebo pomocí programu GhostScript.
%%%
%%%  Rovněž je možné použít program DVIPDFM a vytvořit z dokumentu
%%%  soubor ve formátu PDF včetně hypertextových odkazů.

\subsection{Přepínače}
Styl kidiplom je z hlediska uživatele zastoupen ekvivalentně nazvanou třídou, kterou je třeba volat na záčátku dokumentu:
\begin{kicode}{TeX}{}{Volání třídy \textbf{kidiplom}}
\documentclass[false
  master=true,
  font=sans,
  printversion=false,
  joinlists=true,
  glossaries=true,
  figures=true,
  tables=true,
  sourcecodes=true,
  theorems=true,
  bibencoding=utf8,
  language=czech,
  encoding=utf8,
  field=inf,
  index=true,
  biblatex=true
]{kidiplom}
\end{kicode}

Následuje přehled přepínačů, je vždy uvedeno jméno přepínač, včetně výchozí hodnoty. Přepínače uvádí tabulka \ref{tab:prepinace}.

\begin{table}
\begin{center}
\caption{Seznam přepínačů}\label{tab:prepinace}
\scalebox{0.95}{\begin{tabular}{>{\bfseries}l >{\ttfamily}c L{8cm}}
{\normalfont Přepínač} & {\normalfont Výchozí hodnota} & {\normalfont Popis} \\
\hline
master & false & Povolí nebo zakáže režim diplomové práce. Výchozí režim je tedy bakalářská práce. \\

field & ainfp & Specifikuje studijní obor:\newline
\begin{description}
\item[ainf] Aplikovaná informatika\,--\,prezenční,
\item[ainfk] Aplikovaná informatika\,--\,kombinovaná,
\item[inf] Informatika\,--\,prezenční,
\item[infv] Informatika ve vzdělávání\,--\,kombinovaná,
\item[binf] Bioinformatika\,--\,prezenční.
\end{description} \\

font & serif & Zapne či vypne podporu pěkného bezpatkového fontu. Možné hodnoty jsou:\newline
\begin{description}
\item[sans] Bezpatkové písmo (písmo Iwona).
\item[serif] Patkové písmo (písmo Computer Modern).
\end{description} \\

%%  'encoding=kódování' pro kódování tohoto a vložených zdrojových
%%  textů v kódování jiném než výchozím utf8
encoding & utf8 & Kódování souboru dokumentu, doporučuje se ponechat výchozí hodnotu. \\

bibencoding & utf8 & Kódování souboru bibliografie. Tato volba má smysl pouze, pokud je použita bibliografie skrze balíček \BibLaTeX{}. \\

language & czech & Jazyk práce. \\

printversion & false & Je-li zapnuto, pak budou odkazy vysázeny optimalizovaně pro knižní sazbu. Tuto volbu je nutno použít pro tisk práce. \\

%%% Nepovinné argumenty `tables' a `figures' použijte pouze v případě,
%%% že váš dokument obsahuje tabulky a obrázky a chcete vytvořit
%%% jejich seznamy za obsahem.
%%%
%%% Argument `joinlists' způsobí zřetězení obsahu a seznamů tabulek a obrázků.
%%% Není-li použít, všechny seznamy jsou uvedeny na samostatných stránkách.

joinlists & true & Je-li zapnuto, pak seznamy obrázků, tabulek, vět a
zdrojových kódů sázené za obsahem nebudou rozděleny na samostatné stránky. \\

figures & true & Je-li zapnuto, pak v seznamech položek bude zahrnut seznam obrázků. \\

tables & true & Je-li zapnuto, pak v seznamech položek bude zahrnut seznam tabulek. \\

theorems & false & Je-li zapnuto, pak v seznamech bude zahrnut seznam teorémů. \\

sourcecodes & false & Je-li zapnuto, pak v seznamech bude zahrnut seznam zdrojových kódů. \\

glossaries & false & Je-li zapnuto, pak na konci dokumentu bude vysázen seznam zkratek. \\

index & false & Zapíná podporu sazby rejstříku. \\

biblatex & true & Zapne sazbu bibliografie přes balík \BibLaTeX{}.
\end{tabular}}
\end{center}
\end{table}

\subsection{Geometrie stránky}
Tento styl používá list velikosti $A4$. Pro sazbu prací je třeba použít jednostrannou sazbu. Levý okraj je rozšířen s ohledem na vazbu výsledné knižní podoby práce.

\section{Sazba částí dokumentu}
\subsection{Sazba úvodní strany či obsahu}
Vysázení všech podstatných částí úvodu práce obstará makro \kiinlinecode{TeX}{!}{\\maketitle}. Pro správné vysázení všech částí a meta-informací je potřeba použí makra \kiinlinecode{TeX}{!}{\\title}, \kiinlinecode{TeX}{!}{\\author} a další. Jejich přehled lze najít ve zdrojovém souboru tohoto dokumentu. V případě použítí \textbf{pdf} výstupu se generuje i dodatečná hlavička souboru s meta-informacemi jako je autor dokumentu, název práce či dalšími.

\subsection{Závěry}
Závěr práce by se měl poskytnout jak v původním jazyce práce, tak v jazyce anglickém. Pro sazbu závěru jsou k dispozici příslušná makra. Berte na vědomí, že v anglickém závěru se aktivuje plně anglická sazba se všemi konvencemi. Tedy je třeba používat anglické uvozovky a další správné typografické prvky.

\begin{kicode}{TeX}{}{Sazba závěrů}
% Tiskne český závěr práce.
\begin{kiconclusions}
Závěr práce v \uv{českém} jazyce.
\end{kiconclusions}

% Tiskne anglický závěr práce.
\begin{kiconclusions}[english]
Thesis conclusions written in \uv{English}.
\end{kiconclusions}
\end{kicode}

\subsection{Matematika}
Pro sazbu matematiky je k dispozici sada standardních maker.
$$\langle f \rangle, \lfloor g \rfloor,
\lceil h \rceil, \ulcorner i \urcorner$$

$$\left\{\frac{x^2}{y^3}\right\}$$

$$
A_{m,n} =
 \begin{pmatrix}
  a_{1,1} & a_{1,2} & \cdots & a_{1,n} \\
  a_{2,1} & a_{2,2} & \cdots & a_{2,n} \\
  \vdots  & \vdots  & \ddots & \vdots  \\
  a_{m,1} & a_{m,2} & \cdots & a_{m,n}
 \end{pmatrix}
$$

$$
M = \begin{bmatrix}
       \frac{5}{6} & \frac{1}{6} & 0           \\[0.3em]
       \frac{5}{6} & 0           & \frac{1}{6} \\[0.3em]
       0           & \frac{5}{6} & \frac{1}{6}
     \end{bmatrix}
$$

\subsection{Sazba literatury}
Pro sazbu literatury má uživatel dvě možnosti. Může použít služeb balíků \BibLaTeX{}, který je pro \textbf{kistyles} zapnutý, či lze použít manuální sazbu bibliografie.
\subsubsection{Sazba bibliografie přes \BibLaTeX{}}
Při použití tohoto balíku se data o použité literatuře ukládají do dedikovaného textového souboru, ukázku najdete i v tomto stylu pod jménem \kiinlinecode{text}{!}{bibliografie.bib}.

Formát daného souboru je nad rámec této dokumentace a je na každém uživateli, aby si jej nastudoval. Bibliografie se tiskne makrem \kiinlinecode{TeX}{!}{\\printbibliography}. Taktéž v preambuli dokumentu je třeba definovat, který soubor data bibliografie obsahuje, tedy například \kiinlinecode{TeX}{!}{\\bibliography\{bibliografie.bib\}}.

Dokument, který využívá \BibLaTeX{} je následně nutné přeložit jak pomocí překladače zvoleného ovladače, tak pomocí aplikace \kiinlinecode{text}{!}{biber}. Více informací poskytne soubor \kiinlinecode{text}{!}{Makefile} z distribuce tohoto stylu.

Výhodou tohoto přístupu je, že bibliografie se vysází automaticky a (obvykle) není třeba manuální úprava formátování.

\subsubsection{Manuální sazba bibliografie}
Manuální sazba obnáší vysázení prostředí \kiinlinecode{text}{!}{thebibliography} ručně. To je nad rámec tohoto dokumentu. Ukázku tohoto přístupu lze samozřejmě nalézt ve zdrojovém souboru tohoto dokumentu nebo také \href{http://www.math.uiuc.edu/~hildebr/tex/bibliographies.html}{zde}.

Pro aktivaci manuální sazby bibliografie je třeba volat třídu \kiinlinecode{text}{!}{kidiplom} s parametrem \kiinlinecode{text}{!}{biblatex=false}. Mějte, prosím, na paměti, že v tomto módu jsou makra \kiinlinecode{text}{!}{\\bibliography} a \kiinlinecode{text}{!}{\\printbibliography} nedostupná.

\subsection{Drobná makra}
Základní styl definuje hned několik maker pro usnadnění práce. Například makro \kiinlinecode{TeX}{!}{\\buno} vysází řetezec \uv{bez újmy na obecnosti}. Je k dispozici i verze s prvním velkým písmenem, \kiinlinecode{TeX}{!}{\\Buno}.

Je rovněž možno přidávat položky do seznamu zkratek. K tomu slouží makro \kiinlinecode{TeX}{!}{\\newacronym}, které lze použít například jednoduše jako \kiinlinecode{TeX}{!}{\\newacronym\{UPOL\}\{UPOL\}\{\\kitextunivcz\}}. Na danou zkratku se pak lze odkazovat jednoduše, \kiinlinecode{TeX}{!}{\\gls\{UPOL\}}.

Sazba uvozovek respektuje nastavení částí dokumentu, a proto se doporučuje používat makro \kiinlinecode{TeX}{!}{\\uv}. V anglické závěru práce toto platí taky, viz tato PDF ukázka.

Styl podporuje sazbu odstavců v tabulkách, více obsahuje tabulka \ref{tab:odstavce}.

\begin{table}
\begin{center}
\caption{Seznam přepínačů}\label{tab:odstavce}
\begin{tabular}{L{4cm}|R{4cm}|L{4cm}}
\lipsum[23] & \lipsum[22] & \lipsum[21]
\end{tabular}
\end{center}
\end{table}

K dispozici jsou také makra pro sazbu \csharp{} (\kiinlinecode{TeX}{!}{\\csharp}) či \cpp{} (\kiinlinecode{TeX}{!}{\\cpp}).

%% v případě tvorby rejstříku přeložit vygenerovaný soubor .idx
%% programem Makeindex a v případě tvorby seznamu zkratek spustit
%% program Makeglossaries s parametrem jméno souboru zdrojového textu
%% bez přípony a následně opět (dvakrát) přeložit zdrojový text
%% programem pdfLaTeX.

\subsection{Sazba rejstříku}
Sazba rejstříku sestává z několika kroků:

\begin{enumerate}
\item Je třeba přes volbu \kiinlinecode{TeX}{!}{index=true} rejstříkování povolit.
\item Použítím makra \kiinlinecode{TeX}{!}{\\index} rejstříkovat vybrané pojmy.
\item Kompilovat s použitím utility \kiinlinecode{TeX}{!}{makeindex}. Pro specifika tohoto kroku si stačí prohlédnout soubor \kiinlinecode{text}{!}{Makefile}.
\end{enumerate}

Makro \kiinlinecode{TeX}{!}{\\index} je redefinováno tak, že sází klikací odkaz na výraz v rejstříku. Je doporučeno jej použít ihned za výrazem\index{výraz}.

\textbf{Omezení redefinovaného makra \kiinlinecode{TeX}{!}{\\index}}: klikací odkaz nefunguje, pokud použijete konstrukci \kiinlinecode{TeX}{!}{\\index\{výraz|makro\}} (resp. \kiinlinecode{TeX}{!}{\\index\{výraz|(makro\}}), např. \kiinlinecode{TeX}{!}{\\index\{výraz|textit\}}.

Rejstřík lze vysázet pomocí makra \kiinlinecode{TeX}{!}{\\printindex}.

\subsection{Sazba zdrojových kódů}
Styl nabízí dva způsoby sazby zdrojových kódů:

\begin{enumerate}
\item Sazbu řádkových kódů, například \kiinlinecode{CSS}{!}{background-color: white;}. K tomu slouží makro formátu \kiinlinecode{TeX}{!}{\\kiinlinecode\{jazyk\}\{separátor\}\{kód\}}. Za separátor je vhodné volit jakýkoliv znak, který se nevyskytuje v samotném sázeném zdrojovém kódu. Za jazyk je nutno dosadit jeden z těchto: C, TeX, PHP, HTML, Lisp, SQL, TeX, Python, Java, TutorialD, text, csharp, cpp, JavaScript, CSS.

\item Sazbu zdrojových kódu do separátních prostředí. Takto vytištěný kód se objeví v seznamu zdrojových kódů. Ukázka například zdrojový kód \ref{kod:cpp}. Ukázku sazby naleznete ve zdrojovém kódu tohoto dokumentu.
\end{enumerate}

\newacronym{UPOL}{UPOL}{\kitextunivcz}

\begin{definition}[Název definice]
Abcd. Abcd. Abcd. Abcd. Abcd. Abcd. Abcd. Abcd. Abcd. Abcd. Abcd. Abcd. Abcd. Abcd. Abcd. Abcd. Abcd. Abcd. Abcd. Abcd. Abcd. Abcd. Abcd. Abcd. Abcd. Abcd. Abcd. Abcd. Abcd. Abcd. \gls{UPOL}
\end{definition}

\begin{proof}[Název důkazu]
Abcd. Abcd. Abcd. Abcd. Abcd. Abcd. Abcd. Abcd. Abcd. Abcd. Abcd. Abcd. Abcd. Abcd. Abcd. Abcd. Abcd. Abcd. Abcd. Abcd. Abcd. Abcd. Abcd. Abcd. Abcd. Abcd. Abcd. Abcd. Abcd. Abcd. 
\end{proof}

\begin{remark}[Pumpovací věta]
Abcd. Abcd. Abcd. Abcd. Abcd. Abcd. Abcd. Abcd. Abcd. Abcd. Abcd. Abcd. Abcd. Abcd. Abcd. Abcd. Abcd. Abcd. Abcd. Abcd. Abcd. Abcd. Abcd. Abcd. Abcd. Abcd. Abcd. Abcd. Abcd. Abcd. 
\end{remark}

\begin{example}[Pumpovací věta]
Abcd. Abcd. Abcd. Abcd. Abcd. Abcd. Abcd. Abcd. Abcd. Abcd. Abcd. Abcd. Abcd. Abcd. Abcd. Abcd. Abcd. Abcd. Abcd. Abcd. Abcd. Abcd. Abcd. Abcd. Abcd. Abcd. Abcd. Abcd. Abcd. Abcd. 
\end{example}

\begin{lemma}[Název definice]
Abcd. Abcd. Abcd. Abcd. Abcd. Abcd. Abcd. Abcd. Abcd. Abcd. Abcd. Abcd. Abcd. Abcd. Abcd. Abcd. Abcd. Abcd. Abcd. Abcd. Abcd. Abcd. Abcd. Abcd. Abcd. Abcd. Abcd. Abcd. Abcd. Abcd. 
\end{lemma}

\begin{consequence}[Název důkazu]
Abcd. Abcd. Abcd. Abcd. Abcd. Abcd. Abcd. Abcd. Abcd. Abcd. Abcd. Abcd. Abcd. Abcd. Abcd. Abcd. Abcd. Abcd. Abcd. Abcd. Abcd. Abcd. Abcd. Abcd. Abcd. Abcd. Abcd. Abcd. Abcd. 
\end{consequence}

\begin{theorem}[Pumpovací věta]
Abcd. Abcd. Abcd. Abcd. Abcd. Abcd. Abcd. Abcd. Abcd. Abcd. Abcd. Abcd. Abcd. Abcd. Abcd. Abcd. Abcd. Abcd. Abcd. Abcd. Abcd. Abcd. Abcd. Abcd. Abcd. Abcd. Abcd. Abcd. Abcd. Abcd. 
\end{theorem}


\begin{kicode}{cpp}{kod:cpp}{\cpp}
int main("cs acsa") // komentar
int main("cs acsa") // komentar
int main("cs acsa") // komentar
int main("cs acsa") // komentar
int main("cs acsa") // komentar
\end{kicode}

\begin{kicode}{JavaScript}{}{JS}
new object() // komentar
\end{kicode}

\begin{kicode}{csharp}{}{\csharp}
public static int main("cs acsa") // komentar
\end{kicode}

\begin{kicode}{SQL}{}{SQL}
SELECT * FROM table_1; /* komentar */
\end{kicode}

\begin{kicode}{TutorialD}{}{TutorialD}
table_1 AND table_2;
\end{kicode}

%% Závěry práce. V jazyce práce a anglicky. Text pro jiný než
%% nastavený jazyk práce (nepovinným parametrem language makra
%% \documentclass, výchozí český) se zadává použitím makra s uvedením
%% jazyka jako nepovinného parametru.
\begin{kiconclusions}
Závěr práce v \uv{českém} jazyce.
\end{kiconclusions}

\begin{kiconclusions}[english]
Thesis conclusions in \uv{English}.
\end{kiconclusions}

%% Přílohy obsahu textu práce, za makrem \appendix.
\appendix

\section{První příloha}
Text první přílohy

\section{Druhá příloha}
Text druhé přílohy

%% Obsah přiloženého CD/DVD. Poslední příloha. Upravte podle vlastní
%% práce!
\section{Obsah přiloženého CD/DVD} \label{sec:ObsahCD}

Na samotném konci textu práce je uveden stručný popis obsahu
přiloženého CD/DVD, tj.~jeho závazné adresářové struktury, důležitých
souborů apod.

\begin{description}

\item[\texttt{bin/}] \hfill \\
  Instalátor \textsc{Instalator} programu, popř.~program
  \textsc{Program}, spustitelné přímo z~CD/DVD. / Kompletní adresářová
  struktura webové aplikace \textsc{Webovka} (v~ZIP archivu) pro
  zkopírování na webový server. Adresář obsahuje i~všechny runtime
  knihovny a~další soubory potřebné pro bezproblémový běh instalátoru
  a~programu z~CD/DVD / pro bezproblémový provoz webové aplikace na
  webovém serveru.

\item[\texttt{doc/}] \hfill \\
  Text práce ve formátu PDF, vytvořený s~použitím závazného stylu KI
  PřF UP v~Olomouci pro závěrečné práce, včetně všech příloh,
  a~všechny soubory potřebné pro bezproblémové vygenerování PDF
  dokumentu textu (v~ZIP archivu), tj.~zdrojový text textu, vložené
  obrázky, apod.

\item[\texttt{src/}] \hfill \\
  Kompletní zdrojové texty programu \textsc{Program} / webové aplikace
  \textsc{Webovka} se všemi potřebnými (příp.~převzatými) zdrojovými
  texty, knihovnami a~dalšími soubory potřebnými pro bezproblémové
  vytvoření spustitelných verzí programu / adresářové struktury pro
  zkopírování na webový server.

\item[\texttt{readme.txt}] \hfill \\
  Instrukce pro instalaci a~spuštění programu \textsc{Program}, včetně
  všech požadavků pro jeho bezproblémový provoz. / Instrukce pro
  nasazení webové aplikace \textsc{Webovka} na webový server, včetně
  všech požadavků pro její bezproblémový provoz, a~webová adresa, na
  které je aplikace nasazena pro účel testování při tvorbě posudků
  práce a~pro účel obhajoby práce.

\end{description}

Navíc CD/DVD obsahuje:

\begin{description}

\item[\texttt{data/}] \hfill \\
  Ukázková a~testovací data použitá v~práci a~pro potřeby testování
  práce při tvorbě posudků a~obhajoby práce.

\item[\texttt{install/}] \hfill \\
  Instalátory aplikací, runtime knihoven a~jiných souborů potřebných
  pro provoz programu \textsc{Program} / webové aplikace
  \textsc{Webovka}, které nejsou standardní součástí operačního
  systému určeného pro běh programu / provoz webové aplikace.

\item[\texttt{literature/}] \hfill \\
  Vybrané položky bibliografie, příp.~jiná užitečná literatura
  vztahující se k~práci.

\end{description}

U~veškerých cizích převzatých materiálů obsažených na CD/DVD jejich
zahrnutí dovolují podmínky pro jejich šíření nebo přiložený souhlas
držitele copyrightu. Pro všechny použité (a~citované) materiály,
u~kterých toto není splněno a~nejsou tak obsaženy na CD/DVD, je uveden
jejich zdroj (např.~webová adresa) v~bibliografii nebo textu práce
nebo v souboru \texttt{readme.txt}.

%% -------------------------------------------------------------------

%% Sazba volitelného seznamu zkratek, za přílohami.
\printglossary

%% Sazba povinné bibliografie, za přílohami (případně i za seznamem
%% zkratek). Při použití BibLaTeXu použijte makro
%% \printbibliography. jinak prostředí thebibliography. Ne obojí!

%% Sazba i v textu necitovaných zdrojů, při použití
%% BibLaTeXu. Volitelné.
\nocite{*}
%% Vlastní sazba bibliografie při použití BibLaTeXu.
\printbibliography

%% Bibliografie, včetně sazby, při nepoužití BibLaTeXu.
% \begin{thebibliography}{9}
%\bibitem{kniha2} \uppercase{Hawke}, Paul. NanoHttpd: Light-weight HTTP server designed for embedding in other applications. GitHub [online]. 2014-05-12. [cit. 2014-12-06]. Dostupné z: \url{https://github.com/NanoHttpd/nanohttpd}
%
%\bibitem{jeske13} \uppercase{Jeske}, David; \uppercase{Novák}, Josef. Simple HTTP Server in \csharp: Threaded synchronous HTTP Server abstract class, to respond to HTTP requests. CodeProject: For those who code [online]. 2014-05-24. [cit. 2014-12-06]. Dostupné z: \url{http://www.codeproject.com/Articles/137979/Simple-HTTP-Server-in-C}
%
%\bibitem{uzis2012} \uppercase{ÚSTAV ZDRAVOTNICKÝCH INFORMACÍ A STATISTIKY ČR}. Lékaři, zubní lékaři a farmaceuti 2012 [online]. Praha 2, Palackého náměstí 4: Ústav zdravotnických informací a statistiky ČR, 2012 [cit. 2014-12-06]. ISBN 978-80-7472-089-5. Dostupné z: \url{http://www.uzis.cz/publikace/lekari-zubni-lekari-farmaceuti-2012}
% \end{thebibliography}

%% Sazba volitelného rejstříku, za bibliografií.
\printindex

\end{document}
